\chap{pre}{Einführung}

Dies ist der Umdruck zum \coursename. Diese Arbeitsmappe kann während
des Kurses genutzt werden, um Abbildungen und Schaltungen nachzuschlagen oder etwas nachzulesen.
Später können die Unterlagen als Nachschlagewerk oder zum nochmaligen Nachvollziehen der einzelnen 
Schritte genutzt werden.

Ziel des Kurses ist es, einen ESP8266 mit der Arduino-IDE zu programmieren, verschiedene Sensoren
und Aktoren daran anzuschließen und über das Internet zu steuern. 

\sect{aufb}{Aufbau des Umdruckes}

Dieser Umdruck ist so gesetzt, dass man beim Lesen wichtige Teile oder Abschnitte, die eine besondere Bedeutung 
haben, leicht erkennen kann.

\sect{prelink}{Links und Verweise}
In diesem Umdruck gibt es verschiedene Verweise auf Resourcen im WWW. Ein Verweis auf eine Website sieht wiefolgt aus:

\noindent\fulllink{https://github.com/schlingensiepen/cloudification-esp8266}

Da es in der Regel lästig und fehleranfällig ist, einen solchen URL abzutippen, ist in der Regel noch ein Kurzlink mit 
angegeben:

\noindent\shortlink{drj}

Diese Kurzlinks können über die Seite \fulllink{http://link.i3cm.de} angezeigt oder direkt aufgerufen werden.

Wenn Internetseiten als Quelle angegeben werden, ist das wie folgt notiert:

\noindent\linkreference{https://www.arduino.cc/}

Zu diesem Kurs gibt es Beispielquellcode, der über 
\doublelink{https://github.com/schlingensiepen/cloudification-esp8266}{drj}
heruntergeladen werden kann. Laden Sie die kompletten Quellen als zip-Archiv
herunter und entpacken dieses auf der lokalen Festplatte.

Verweise auf eine Beispieldatei aus diesem Archiv werden wie folgt  dargestellt:

\noindent\gitHub{myBlink.ino}

\sect{prekey}{Menüeinträge und Tastenkombinationen}

Anweisungen zum Wählen einer Funktion über das Funktionsmenü einer Softwareanwendung sind wie folgt notiert:

\noindent\menuiii{Hauptmenü}{Menüeintrag}{Unterpunkt}

Die Pfeilrichtung gibt an, in welcher Reihenfolge die Punkte mit der Maus ausgewählt werden müssen.

Tastenkombinationen zum Wählen einer Funktion einer Softwareanwendung über die Tastatur sind wie folgt notiert:

\noindent\keyii{Taste1}{Taste2}

Die Tasten sind in der Regel gleichzeitig zu drücken um die Funktionen aufzurufen.

Im Rahmen des Kurses werden Softwarebibliotheken aus den Arduino-Repositories verwendet. Um die Auswahl der 
richtigen Bibliothek zu erleichtern, ist diese immer mit Name und Autor angegeben:

\noindent\lib{TelegramBotClient}{Jörn Schlingensiepen}

\vfill\null\pagebreak
\sect{precode}{Quellcode}

Zur Veranschaulichung sind Beispielquelltexte eingebunden. Diese werden wie folgt dargestellt:

\begin{src}
  int fak(int i)
  {
    if (i==0) return 1;
    return fak(i-1);
  }
\end{src}

Wird im Fließtext auf Quellcode Bezug genommen, so wird dieser innerhalb des 
Textes durch Verwendung einer anderen Schriftart gekennzeichnet. Aus dem
Beispiel oben könnte so \code{int fak(int i)} im Fließtext dargestellt werden.

Neben Quellcodes werden auch Flussdiagramme verwendet. Ein Bezug im Text zu einem bestimmten
Schritt im Flussdiagramm wird so dargestellt: \step{2}.

\sect{predef}{Begriffe und Definitionen}

Wenn im Fließtext ein Begriff eingeführt wird, der zur Fachsprache gehört und deshalb
erinnernswert ist, wird dieser wie folgt hervorgehoben:

\noindent\term{Serielle Schnittstelle}

Wird eine Abkürzung eingeführt sind die Buchstaben, die diese Abkürzung ergeben, wie folgt kenntlich gemacht:

\noindent\abbr{E}lectrically \abbr{E}rasable \abbr{P}rogrammable \abbr{R}ead-\abbr{O}nly \abbr{M}emory kurz \abbr{EEPROM}

Für besonders wichtige Begriffe gibt es Definitionen, die wie folgt dargestellt werden:

\begin{definition}{schueler}{Schüler}
Ein Schüler oder eine Schülerin ist nach der Wortbedeutung ursprünglich eine Person, die im organisatorischen Rahmen einer Schule lernt. 
Dabei erhalten sie von entsprechenden Lehrern Unterricht, um in einem Lernprozess Fertigkeiten, Wissen, Einsichten und Verhaltensweisen zu erwerben, 
zu erweitern, einzuüben und zu vertiefen. 
\end{definition}

\sect{preinfo}{Beigefügte Informationen}

Zu einigen Themen gibt es Exkurse, die wie folgt dargestellt sind:
\end{multicols}
\begin{excursus}{pre}{Bedeutung des Exkurses}
Exkurse behandeln Themen, die beim Verständnis des Kurse helfen oder interessantes Hintergrundwissen darstellen. Grundsätzlich kann der Kurs auch
ohne das Lesen der Exkurse bewältigt werden.
\end{excursus}
\begin{multicols}{2}

Zum Nachvollziehen und Vertiefen des Stoffes sind Aufgabenstellungen in den Kurs eingearbeitet, diese sind wie folgt dargestellt:

\begin{excercise}{pre}{Aufgabentitel}
Beschreibung der Aufgabenstellung 
\end{excercise}

Zusätzliche besondere Hinweise sind wie folgt dargestellt:

\begin{anotation}
Besonderer Hinweis.
\end{anotation}



\setcounter{excursuscounter}{0}
\setcounter{definitioncounter}{0}
\setcounter{excercisecounter}{0}
